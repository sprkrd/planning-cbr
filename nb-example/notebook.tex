
% Default to the notebook output style

    


% Inherit from the specified cell style.




    
\documentclass[11pt]{article}

    
    
    \usepackage[T1]{fontenc}
    % Nicer default font (+ math font) than Computer Modern for most use cases
    \usepackage{mathpazo}

    % Basic figure setup, for now with no caption control since it's done
    % automatically by Pandoc (which extracts ![](path) syntax from Markdown).
    \usepackage{graphicx}
    % We will generate all images so they have a width \maxwidth. This means
    % that they will get their normal width if they fit onto the page, but
    % are scaled down if they would overflow the margins.
    \makeatletter
    \def\maxwidth{\ifdim\Gin@nat@width>\linewidth\linewidth
    \else\Gin@nat@width\fi}
    \makeatother
    \let\Oldincludegraphics\includegraphics
    % Set max figure width to be 80% of text width, for now hardcoded.
    \renewcommand{\includegraphics}[1]{\Oldincludegraphics[width=.8\maxwidth]{#1}}
    % Ensure that by default, figures have no caption (until we provide a
    % proper Figure object with a Caption API and a way to capture that
    % in the conversion process - todo).
    \usepackage{caption}
    \DeclareCaptionLabelFormat{nolabel}{}
    \captionsetup{labelformat=nolabel}

    \usepackage{adjustbox} % Used to constrain images to a maximum size 
    \usepackage{xcolor} % Allow colors to be defined
    \usepackage{enumerate} % Needed for markdown enumerations to work
    \usepackage{geometry} % Used to adjust the document margins
    \usepackage{amsmath} % Equations
    \usepackage{amssymb} % Equations
    \usepackage{textcomp} % defines textquotesingle
    % Hack from http://tex.stackexchange.com/a/47451/13684:
    \AtBeginDocument{%
        \def\PYZsq{\textquotesingle}% Upright quotes in Pygmentized code
    }
    \usepackage{upquote} % Upright quotes for verbatim code
    \usepackage{eurosym} % defines \euro
    \usepackage[mathletters]{ucs} % Extended unicode (utf-8) support
    \usepackage[utf8x]{inputenc} % Allow utf-8 characters in the tex document
    \usepackage{fancyvrb} % verbatim replacement that allows latex
    \usepackage{grffile} % extends the file name processing of package graphics 
                         % to support a larger range 
    % The hyperref package gives us a pdf with properly built
    % internal navigation ('pdf bookmarks' for the table of contents,
    % internal cross-reference links, web links for URLs, etc.)
    \usepackage{hyperref}
    \usepackage{longtable} % longtable support required by pandoc >1.10
    \usepackage{booktabs}  % table support for pandoc > 1.12.2
    \usepackage[inline]{enumitem} % IRkernel/repr support (it uses the enumerate* environment)
    \usepackage[normalem]{ulem} % ulem is needed to support strikethroughs (\sout)
                                % normalem makes italics be italics, not underlines
    

    
    
    % Colors for the hyperref package
    \definecolor{urlcolor}{rgb}{0,.145,.698}
    \definecolor{linkcolor}{rgb}{.71,0.21,0.01}
    \definecolor{citecolor}{rgb}{.12,.54,.11}

    % ANSI colors
    \definecolor{ansi-black}{HTML}{3E424D}
    \definecolor{ansi-black-intense}{HTML}{282C36}
    \definecolor{ansi-red}{HTML}{E75C58}
    \definecolor{ansi-red-intense}{HTML}{B22B31}
    \definecolor{ansi-green}{HTML}{00A250}
    \definecolor{ansi-green-intense}{HTML}{007427}
    \definecolor{ansi-yellow}{HTML}{DDB62B}
    \definecolor{ansi-yellow-intense}{HTML}{B27D12}
    \definecolor{ansi-blue}{HTML}{208FFB}
    \definecolor{ansi-blue-intense}{HTML}{0065CA}
    \definecolor{ansi-magenta}{HTML}{D160C4}
    \definecolor{ansi-magenta-intense}{HTML}{A03196}
    \definecolor{ansi-cyan}{HTML}{60C6C8}
    \definecolor{ansi-cyan-intense}{HTML}{258F8F}
    \definecolor{ansi-white}{HTML}{C5C1B4}
    \definecolor{ansi-white-intense}{HTML}{A1A6B2}

    % commands and environments needed by pandoc snippets
    % extracted from the output of `pandoc -s`
    \providecommand{\tightlist}{%
      \setlength{\itemsep}{0pt}\setlength{\parskip}{0pt}}
    \DefineVerbatimEnvironment{Highlighting}{Verbatim}{commandchars=\\\{\}}
    % Add ',fontsize=\small' for more characters per line
    \newenvironment{Shaded}{}{}
    \newcommand{\KeywordTok}[1]{\textcolor[rgb]{0.00,0.44,0.13}{\textbf{{#1}}}}
    \newcommand{\DataTypeTok}[1]{\textcolor[rgb]{0.56,0.13,0.00}{{#1}}}
    \newcommand{\DecValTok}[1]{\textcolor[rgb]{0.25,0.63,0.44}{{#1}}}
    \newcommand{\BaseNTok}[1]{\textcolor[rgb]{0.25,0.63,0.44}{{#1}}}
    \newcommand{\FloatTok}[1]{\textcolor[rgb]{0.25,0.63,0.44}{{#1}}}
    \newcommand{\CharTok}[1]{\textcolor[rgb]{0.25,0.44,0.63}{{#1}}}
    \newcommand{\StringTok}[1]{\textcolor[rgb]{0.25,0.44,0.63}{{#1}}}
    \newcommand{\CommentTok}[1]{\textcolor[rgb]{0.38,0.63,0.69}{\textit{{#1}}}}
    \newcommand{\OtherTok}[1]{\textcolor[rgb]{0.00,0.44,0.13}{{#1}}}
    \newcommand{\AlertTok}[1]{\textcolor[rgb]{1.00,0.00,0.00}{\textbf{{#1}}}}
    \newcommand{\FunctionTok}[1]{\textcolor[rgb]{0.02,0.16,0.49}{{#1}}}
    \newcommand{\RegionMarkerTok}[1]{{#1}}
    \newcommand{\ErrorTok}[1]{\textcolor[rgb]{1.00,0.00,0.00}{\textbf{{#1}}}}
    \newcommand{\NormalTok}[1]{{#1}}
    
    % Additional commands for more recent versions of Pandoc
    \newcommand{\ConstantTok}[1]{\textcolor[rgb]{0.53,0.00,0.00}{{#1}}}
    \newcommand{\SpecialCharTok}[1]{\textcolor[rgb]{0.25,0.44,0.63}{{#1}}}
    \newcommand{\VerbatimStringTok}[1]{\textcolor[rgb]{0.25,0.44,0.63}{{#1}}}
    \newcommand{\SpecialStringTok}[1]{\textcolor[rgb]{0.73,0.40,0.53}{{#1}}}
    \newcommand{\ImportTok}[1]{{#1}}
    \newcommand{\DocumentationTok}[1]{\textcolor[rgb]{0.73,0.13,0.13}{\textit{{#1}}}}
    \newcommand{\AnnotationTok}[1]{\textcolor[rgb]{0.38,0.63,0.69}{\textbf{\textit{{#1}}}}}
    \newcommand{\CommentVarTok}[1]{\textcolor[rgb]{0.38,0.63,0.69}{\textbf{\textit{{#1}}}}}
    \newcommand{\VariableTok}[1]{\textcolor[rgb]{0.10,0.09,0.49}{{#1}}}
    \newcommand{\ControlFlowTok}[1]{\textcolor[rgb]{0.00,0.44,0.13}{\textbf{{#1}}}}
    \newcommand{\OperatorTok}[1]{\textcolor[rgb]{0.40,0.40,0.40}{{#1}}}
    \newcommand{\BuiltInTok}[1]{{#1}}
    \newcommand{\ExtensionTok}[1]{{#1}}
    \newcommand{\PreprocessorTok}[1]{\textcolor[rgb]{0.74,0.48,0.00}{{#1}}}
    \newcommand{\AttributeTok}[1]{\textcolor[rgb]{0.49,0.56,0.16}{{#1}}}
    \newcommand{\InformationTok}[1]{\textcolor[rgb]{0.38,0.63,0.69}{\textbf{\textit{{#1}}}}}
    \newcommand{\WarningTok}[1]{\textcolor[rgb]{0.38,0.63,0.69}{\textbf{\textit{{#1}}}}}
    
    
    % Define a nice break command that doesn't care if a line doesn't already
    % exist.
    \def\br{\hspace*{\fill} \\* }
    % Math Jax compatability definitions
    \def\gt{>}
    \def\lt{<}
    % Document parameters
    \title{final\_presentation}
    
    
    

    % Pygments definitions
    
\makeatletter
\def\PY@reset{\let\PY@it=\relax \let\PY@bf=\relax%
    \let\PY@ul=\relax \let\PY@tc=\relax%
    \let\PY@bc=\relax \let\PY@ff=\relax}
\def\PY@tok#1{\csname PY@tok@#1\endcsname}
\def\PY@toks#1+{\ifx\relax#1\empty\else%
    \PY@tok{#1}\expandafter\PY@toks\fi}
\def\PY@do#1{\PY@bc{\PY@tc{\PY@ul{%
    \PY@it{\PY@bf{\PY@ff{#1}}}}}}}
\def\PY#1#2{\PY@reset\PY@toks#1+\relax+\PY@do{#2}}

\expandafter\def\csname PY@tok@gd\endcsname{\def\PY@tc##1{\textcolor[rgb]{0.63,0.00,0.00}{##1}}}
\expandafter\def\csname PY@tok@sc\endcsname{\def\PY@tc##1{\textcolor[rgb]{0.73,0.13,0.13}{##1}}}
\expandafter\def\csname PY@tok@sx\endcsname{\def\PY@tc##1{\textcolor[rgb]{0.00,0.50,0.00}{##1}}}
\expandafter\def\csname PY@tok@cpf\endcsname{\let\PY@it=\textit\def\PY@tc##1{\textcolor[rgb]{0.25,0.50,0.50}{##1}}}
\expandafter\def\csname PY@tok@s1\endcsname{\def\PY@tc##1{\textcolor[rgb]{0.73,0.13,0.13}{##1}}}
\expandafter\def\csname PY@tok@cs\endcsname{\let\PY@it=\textit\def\PY@tc##1{\textcolor[rgb]{0.25,0.50,0.50}{##1}}}
\expandafter\def\csname PY@tok@mf\endcsname{\def\PY@tc##1{\textcolor[rgb]{0.40,0.40,0.40}{##1}}}
\expandafter\def\csname PY@tok@cm\endcsname{\let\PY@it=\textit\def\PY@tc##1{\textcolor[rgb]{0.25,0.50,0.50}{##1}}}
\expandafter\def\csname PY@tok@vc\endcsname{\def\PY@tc##1{\textcolor[rgb]{0.10,0.09,0.49}{##1}}}
\expandafter\def\csname PY@tok@se\endcsname{\let\PY@bf=\textbf\def\PY@tc##1{\textcolor[rgb]{0.73,0.40,0.13}{##1}}}
\expandafter\def\csname PY@tok@dl\endcsname{\def\PY@tc##1{\textcolor[rgb]{0.73,0.13,0.13}{##1}}}
\expandafter\def\csname PY@tok@c\endcsname{\let\PY@it=\textit\def\PY@tc##1{\textcolor[rgb]{0.25,0.50,0.50}{##1}}}
\expandafter\def\csname PY@tok@nl\endcsname{\def\PY@tc##1{\textcolor[rgb]{0.63,0.63,0.00}{##1}}}
\expandafter\def\csname PY@tok@vg\endcsname{\def\PY@tc##1{\textcolor[rgb]{0.10,0.09,0.49}{##1}}}
\expandafter\def\csname PY@tok@k\endcsname{\let\PY@bf=\textbf\def\PY@tc##1{\textcolor[rgb]{0.00,0.50,0.00}{##1}}}
\expandafter\def\csname PY@tok@na\endcsname{\def\PY@tc##1{\textcolor[rgb]{0.49,0.56,0.16}{##1}}}
\expandafter\def\csname PY@tok@o\endcsname{\def\PY@tc##1{\textcolor[rgb]{0.40,0.40,0.40}{##1}}}
\expandafter\def\csname PY@tok@s2\endcsname{\def\PY@tc##1{\textcolor[rgb]{0.73,0.13,0.13}{##1}}}
\expandafter\def\csname PY@tok@w\endcsname{\def\PY@tc##1{\textcolor[rgb]{0.73,0.73,0.73}{##1}}}
\expandafter\def\csname PY@tok@si\endcsname{\let\PY@bf=\textbf\def\PY@tc##1{\textcolor[rgb]{0.73,0.40,0.53}{##1}}}
\expandafter\def\csname PY@tok@kt\endcsname{\def\PY@tc##1{\textcolor[rgb]{0.69,0.00,0.25}{##1}}}
\expandafter\def\csname PY@tok@nc\endcsname{\let\PY@bf=\textbf\def\PY@tc##1{\textcolor[rgb]{0.00,0.00,1.00}{##1}}}
\expandafter\def\csname PY@tok@kc\endcsname{\let\PY@bf=\textbf\def\PY@tc##1{\textcolor[rgb]{0.00,0.50,0.00}{##1}}}
\expandafter\def\csname PY@tok@kp\endcsname{\def\PY@tc##1{\textcolor[rgb]{0.00,0.50,0.00}{##1}}}
\expandafter\def\csname PY@tok@kd\endcsname{\let\PY@bf=\textbf\def\PY@tc##1{\textcolor[rgb]{0.00,0.50,0.00}{##1}}}
\expandafter\def\csname PY@tok@bp\endcsname{\def\PY@tc##1{\textcolor[rgb]{0.00,0.50,0.00}{##1}}}
\expandafter\def\csname PY@tok@c1\endcsname{\let\PY@it=\textit\def\PY@tc##1{\textcolor[rgb]{0.25,0.50,0.50}{##1}}}
\expandafter\def\csname PY@tok@gu\endcsname{\let\PY@bf=\textbf\def\PY@tc##1{\textcolor[rgb]{0.50,0.00,0.50}{##1}}}
\expandafter\def\csname PY@tok@nd\endcsname{\def\PY@tc##1{\textcolor[rgb]{0.67,0.13,1.00}{##1}}}
\expandafter\def\csname PY@tok@nn\endcsname{\let\PY@bf=\textbf\def\PY@tc##1{\textcolor[rgb]{0.00,0.00,1.00}{##1}}}
\expandafter\def\csname PY@tok@ss\endcsname{\def\PY@tc##1{\textcolor[rgb]{0.10,0.09,0.49}{##1}}}
\expandafter\def\csname PY@tok@m\endcsname{\def\PY@tc##1{\textcolor[rgb]{0.40,0.40,0.40}{##1}}}
\expandafter\def\csname PY@tok@s\endcsname{\def\PY@tc##1{\textcolor[rgb]{0.73,0.13,0.13}{##1}}}
\expandafter\def\csname PY@tok@kr\endcsname{\let\PY@bf=\textbf\def\PY@tc##1{\textcolor[rgb]{0.00,0.50,0.00}{##1}}}
\expandafter\def\csname PY@tok@ne\endcsname{\let\PY@bf=\textbf\def\PY@tc##1{\textcolor[rgb]{0.82,0.25,0.23}{##1}}}
\expandafter\def\csname PY@tok@gt\endcsname{\def\PY@tc##1{\textcolor[rgb]{0.00,0.27,0.87}{##1}}}
\expandafter\def\csname PY@tok@ge\endcsname{\let\PY@it=\textit}
\expandafter\def\csname PY@tok@vi\endcsname{\def\PY@tc##1{\textcolor[rgb]{0.10,0.09,0.49}{##1}}}
\expandafter\def\csname PY@tok@gp\endcsname{\let\PY@bf=\textbf\def\PY@tc##1{\textcolor[rgb]{0.00,0.00,0.50}{##1}}}
\expandafter\def\csname PY@tok@mb\endcsname{\def\PY@tc##1{\textcolor[rgb]{0.40,0.40,0.40}{##1}}}
\expandafter\def\csname PY@tok@mo\endcsname{\def\PY@tc##1{\textcolor[rgb]{0.40,0.40,0.40}{##1}}}
\expandafter\def\csname PY@tok@cp\endcsname{\def\PY@tc##1{\textcolor[rgb]{0.74,0.48,0.00}{##1}}}
\expandafter\def\csname PY@tok@sb\endcsname{\def\PY@tc##1{\textcolor[rgb]{0.73,0.13,0.13}{##1}}}
\expandafter\def\csname PY@tok@kn\endcsname{\let\PY@bf=\textbf\def\PY@tc##1{\textcolor[rgb]{0.00,0.50,0.00}{##1}}}
\expandafter\def\csname PY@tok@vm\endcsname{\def\PY@tc##1{\textcolor[rgb]{0.10,0.09,0.49}{##1}}}
\expandafter\def\csname PY@tok@fm\endcsname{\def\PY@tc##1{\textcolor[rgb]{0.00,0.00,1.00}{##1}}}
\expandafter\def\csname PY@tok@gs\endcsname{\let\PY@bf=\textbf}
\expandafter\def\csname PY@tok@il\endcsname{\def\PY@tc##1{\textcolor[rgb]{0.40,0.40,0.40}{##1}}}
\expandafter\def\csname PY@tok@err\endcsname{\def\PY@bc##1{\setlength{\fboxsep}{0pt}\fcolorbox[rgb]{1.00,0.00,0.00}{1,1,1}{\strut ##1}}}
\expandafter\def\csname PY@tok@ch\endcsname{\let\PY@it=\textit\def\PY@tc##1{\textcolor[rgb]{0.25,0.50,0.50}{##1}}}
\expandafter\def\csname PY@tok@mi\endcsname{\def\PY@tc##1{\textcolor[rgb]{0.40,0.40,0.40}{##1}}}
\expandafter\def\csname PY@tok@sa\endcsname{\def\PY@tc##1{\textcolor[rgb]{0.73,0.13,0.13}{##1}}}
\expandafter\def\csname PY@tok@gr\endcsname{\def\PY@tc##1{\textcolor[rgb]{1.00,0.00,0.00}{##1}}}
\expandafter\def\csname PY@tok@go\endcsname{\def\PY@tc##1{\textcolor[rgb]{0.53,0.53,0.53}{##1}}}
\expandafter\def\csname PY@tok@nb\endcsname{\def\PY@tc##1{\textcolor[rgb]{0.00,0.50,0.00}{##1}}}
\expandafter\def\csname PY@tok@nv\endcsname{\def\PY@tc##1{\textcolor[rgb]{0.10,0.09,0.49}{##1}}}
\expandafter\def\csname PY@tok@sr\endcsname{\def\PY@tc##1{\textcolor[rgb]{0.73,0.40,0.53}{##1}}}
\expandafter\def\csname PY@tok@nf\endcsname{\def\PY@tc##1{\textcolor[rgb]{0.00,0.00,1.00}{##1}}}
\expandafter\def\csname PY@tok@mh\endcsname{\def\PY@tc##1{\textcolor[rgb]{0.40,0.40,0.40}{##1}}}
\expandafter\def\csname PY@tok@sh\endcsname{\def\PY@tc##1{\textcolor[rgb]{0.73,0.13,0.13}{##1}}}
\expandafter\def\csname PY@tok@ni\endcsname{\let\PY@bf=\textbf\def\PY@tc##1{\textcolor[rgb]{0.60,0.60,0.60}{##1}}}
\expandafter\def\csname PY@tok@gh\endcsname{\let\PY@bf=\textbf\def\PY@tc##1{\textcolor[rgb]{0.00,0.00,0.50}{##1}}}
\expandafter\def\csname PY@tok@sd\endcsname{\let\PY@it=\textit\def\PY@tc##1{\textcolor[rgb]{0.73,0.13,0.13}{##1}}}
\expandafter\def\csname PY@tok@ow\endcsname{\let\PY@bf=\textbf\def\PY@tc##1{\textcolor[rgb]{0.67,0.13,1.00}{##1}}}
\expandafter\def\csname PY@tok@nt\endcsname{\let\PY@bf=\textbf\def\PY@tc##1{\textcolor[rgb]{0.00,0.50,0.00}{##1}}}
\expandafter\def\csname PY@tok@no\endcsname{\def\PY@tc##1{\textcolor[rgb]{0.53,0.00,0.00}{##1}}}
\expandafter\def\csname PY@tok@gi\endcsname{\def\PY@tc##1{\textcolor[rgb]{0.00,0.63,0.00}{##1}}}

\def\PYZbs{\char`\\}
\def\PYZus{\char`\_}
\def\PYZob{\char`\{}
\def\PYZcb{\char`\}}
\def\PYZca{\char`\^}
\def\PYZam{\char`\&}
\def\PYZlt{\char`\<}
\def\PYZgt{\char`\>}
\def\PYZsh{\char`\#}
\def\PYZpc{\char`\%}
\def\PYZdl{\char`\$}
\def\PYZhy{\char`\-}
\def\PYZsq{\char`\'}
\def\PYZdq{\char`\"}
\def\PYZti{\char`\~}
% for compatibility with earlier versions
\def\PYZat{@}
\def\PYZlb{[}
\def\PYZrb{]}
\makeatother


    % Exact colors from NB
    \definecolor{incolor}{rgb}{0.0, 0.0, 0.5}
    \definecolor{outcolor}{rgb}{0.545, 0.0, 0.0}



    
    % Prevent overflowing lines due to hard-to-break entities
    \sloppy 
    % Setup hyperref package
    \hypersetup{
      breaklinks=true,  % so long urls are correctly broken across lines
      colorlinks=true,
      urlcolor=urlcolor,
      linkcolor=linkcolor,
      citecolor=citecolor,
      }
    % Slightly bigger margins than the latex defaults
    
    \geometry{verbose,tmargin=1in,bmargin=1in,lmargin=1in,rmargin=1in}
    
    

    \begin{document}
    
    
    \maketitle
    
    

    
    \section{SEL 2017/18 - Practical Work 3 - CBR prototype for
planning}\label{sel-201718---practical-work-3---cbr-prototype-for-planning}

\begin{enumerate}
\def\labelenumi{\arabic{enumi}.}
\item
  Introduction

  \begin{enumerate}
  \def\labelenumii{\arabic{enumii}.}
  \tightlist
  \item
    Basic principles of the CBR engine project. The general ideas of the
    CBR engine implemented, from a technical point of view. That means
    explaining what are the general implementation guidelines of your
    project (which kind of Case Library structure, which kind of case
    structure, etc., without giving more details).
  \item
    Chosen application domain
  \end{enumerate}
\item
  Requirement Analysis of the CBR engine Project. The requirements
  include both the user requirements (main functionalities of the system
  that user needs, i.e., what the system must do?) and the technical
  requirements of the system (maximum time response of the system,
  maximum memory size of the system, etc.).
\item
  Functional Architecture of the CBR engine Project, describing the
  input, the output, the different components of the system and their
  interactions.
\item
  Proposed CBR engine Project solution design:

  \begin{enumerate}
  \def\labelenumii{\arabic{enumii}.}
  \tightlist
  \item
    Case Structure and Case Library structure designed
  \item
    Methods implementing each CBR cycle step (retrieval, adaptation,
    evaluation and learning)
  \end{enumerate}
\item
  Testing and evaluation.
\item
  Discussion of results.
\end{enumerate}

    \subsection{1. Introduction}\label{introduction}

In this document we present the work done for the third delivery of the
subject Supervised and Experimental Learning (SEL), which is about
designing a CBR for planning. A CBR based on the use of several
Heuristics has been developed and used in three different well known
planning domains. A discussion about the results is also provided at the
end of the document.

\subsubsection{1.1. Basic Principles of the CBR engine
project}\label{basic-principles-of-the-cbr-engine-project}

Our aim was to implement a CBR engine which combined several heuristics
in order to give a closer estimation of the optimal length of the plan
to solve one of our domain's problems. In our project, we have
considered that the cost of performing an action is uniform, so that we
will not make any difference between number of states and cost of a
plan.

The general idea of our work is to use the heuristic calculated by the
CBR as the perfect heuristic of any search algorithm used in pathfinding
and graph traversal (IDA*, A*, etc.). We use A*, which guarantees to
find the optimal plan if an only if the found heuristic is admissible,
that is, it is smaller or equal to the perfect heuristic. However, in
our case, we have relaxed the condition of finding the optimal plan, in
order to save computational cost and find a valid path quicker.
Specifically, we do not look at the absolute value of the heuristic, but
at the order that the heuristic produces in the states which are part of
the plan.

The Case Library Structure we have used is \emph{k-d tree}, the reason
is that we work with numerical values which can be organised following
such a structure. For us, each case is a vector of numbers,
specifically, a vector of heuristics.

\subsubsection{1.2. Chosen application
domain}\label{chosen-application-domain}

For this practical work we have chosen three different domains: The
Tower of Hanoi, Blocks World and Elevators. All of them are typical
domains used in Planning. We present the three domains and show examples
of them using blind search methods (BFS and IDS). We also use those
methods as target in order to test our CBR system.

Nevertheless, our project is quite ambitious because we have implemented
it for general purpose. It can be apply to any kind of problem which can
be written using the sintax used in STRIPS.

\paragraph{1.2.1. The Tower of Hanoi}\label{the-tower-of-hanoi}

The Tower of Hanoi is a mathematical game or puzzle. It consists of
three rods and a number of disks of different sizes, which can slide
onto any rod. The puzzle starts with the disks in a neat stack in
ascending order of size on one rod, the smallest at the top, thus making
a conical shape.

The objective of the puzzle is to move the entire stack to another rod,
obeying the following simple rules:

\begin{enumerate}
\def\labelenumi{\arabic{enumi}.}
\tightlist
\item
  Only one disk can be moved at a time.
\item
  Each move consists of taking the upper disk from one of the stacks and
  placing it on top of another stack.
\item
  No disk may be placed on top of a smaller disk.
\end{enumerate}

We have defined the domain in Python and it contains a single operator:
\texttt{move(?what-disk,\ ?from-object,\ ?to-object)}, which moves disk
\texttt{?what} from object \texttt{?from} (either a peg or another disk)
to object \texttt{?to} (also, either a peg or a disk). Our planning
framework can take care of static preconditions. That is, it does not
instantiate the move operator for moving a disk onto a smaller disk) so
it is not necessary to encode static predicates in the state (the
\texttt{smaller(?disk1,?disk2)} predicates that are typically seen in
PDDL Hanoi domains).

We have also implemented a problem generator. The generator can create
problems for any number of disks and pegs. It also allows random initial
and ending configurations. Our first example will be fairly simple and
will have 3 pegs, all the disks in the first peg in the initial state
and all the disks in the third peg in the goal configuration. Although
the goal in this case details the position of every disk, notice that
this does not necessarily has to be the case always. The goal does not
need to be a complete description of the end configuration. It is
perfectly reasonable, and our framework allows it, to consider partial
states (e.g. we want the biggest disk in the third peg, and we do not
care about the rest).

    \begin{Verbatim}[commandchars=\\\{\}]
{\color{incolor}In [{\color{incolor}1}]:} \PY{k+kn}{import} \PY{n+nn}{planning}
        
        \PY{k+kn}{from} \PY{n+nn}{IPython}\PY{n+nn}{.}\PY{n+nn}{display} \PY{k}{import} \PY{n}{display}
\end{Verbatim}


    \begin{Verbatim}[commandchars=\\\{\}]
{\color{incolor}In [{\color{incolor}2}]:} \PY{n}{domain\PYZus{}name} \PY{o}{=} \PY{l+s+s2}{\PYZdq{}}\PY{l+s+s2}{Hanoi}\PY{l+s+s2}{\PYZdq{}}
        \PY{n}{domain} \PY{o}{=} \PY{n}{planning}\PY{o}{.}\PY{n}{domains}\PY{p}{[}\PY{n}{domain\PYZus{}name}\PY{p}{]}
        \PY{n}{problem} \PY{o}{=} \PY{n}{domain}\PY{o}{.}\PY{n}{generate\PYZus{}problem}\PY{p}{(}\PY{l+m+mi}{3}\PY{p}{,} \PY{l+m+mi}{3}\PY{p}{,} \PY{n}{random}\PY{o}{=}\PY{k+kc}{False}\PY{p}{)}
        \PY{n+nb}{print}\PY{p}{(}\PY{n}{domain}\PY{p}{)}
        \PY{c+c1}{\PYZsh{} print(problem)}
        \PY{n+nb}{print}\PY{p}{(}\PY{l+s+s2}{\PYZdq{}}\PY{l+s+s2}{Init:}\PY{l+s+s2}{\PYZdq{}}\PY{p}{)}
        \PY{n}{display}\PY{p}{(}\PY{n}{problem}\PY{o}{.}\PY{n}{init}\PY{p}{(}\PY{p}{)}\PY{p}{)}
        \PY{n+nb}{print}\PY{p}{(}\PY{l+s+s2}{\PYZdq{}}\PY{l+s+s2}{Goal:}\PY{l+s+s2}{\PYZdq{}}\PY{p}{)}
        \PY{n}{display}\PY{p}{(}\PY{n}{problem}\PY{o}{.}\PY{n}{goal}\PY{p}{(}\PY{p}{)}\PY{p}{)}
        \PY{n+nb}{print}\PY{p}{(}\PY{l+s+s2}{\PYZdq{}}\PY{l+s+s2}{Applicable actions in 1st state:}\PY{l+s+s2}{\PYZdq{}}\PY{p}{)}
        \PY{n}{l} \PY{o}{=} \PY{n+nb}{list}\PY{p}{(}\PY{n+nb}{filter}\PY{p}{(}\PY{n}{problem}\PY{o}{.}\PY{n}{init}\PY{p}{(}\PY{p}{)}\PY{o}{.}\PY{n}{can\PYZus{}apply}\PY{p}{,} \PY{n}{problem}\PY{o}{.}\PY{n}{operators}\PY{p}{(}\PY{p}{)}\PY{p}{)}\PY{p}{)}
        \PY{n+nb}{print}\PY{p}{(}\PY{l+s+s2}{\PYZdq{}}\PY{l+s+se}{\PYZbs{}n}\PY{l+s+s2}{\PYZdq{}}\PY{o}{.}\PY{n}{join}\PY{p}{(}\PY{n+nb}{map}\PY{p}{(}\PY{n+nb}{str}\PY{p}{,} \PY{n}{l}\PY{p}{)}\PY{p}{)}\PY{p}{)}
        \PY{c+c1}{\PYZsh{} print(planning.pddl\PYZus{}utility.pddl\PYZus{}actions(l))}
        \PY{c+c1}{\PYZsh{} print(planning.pddl\PYZus{}utility.pddl\PYZus{}domain(domain, problem.operators()))}
        \PY{c+c1}{\PYZsh{} print(planning.pddl\PYZus{}utility.pddl\PYZus{}problem(problem))}
        
        \PY{k}{with} \PY{n+nb}{open}\PY{p}{(}\PY{l+s+s2}{\PYZdq{}}\PY{l+s+s2}{domain}\PY{l+s+s2}{\PYZdq{}}\PY{o}{+}\PY{n}{domain\PYZus{}name}\PY{o}{+}\PY{l+s+s2}{\PYZdq{}}\PY{l+s+s2}{.pddl}\PY{l+s+s2}{\PYZdq{}}\PY{p}{,} \PY{l+s+s2}{\PYZdq{}}\PY{l+s+s2}{w}\PY{l+s+s2}{\PYZdq{}}\PY{p}{)} \PY{k}{as} \PY{n}{d}\PY{p}{,} \PY{n+nb}{open}\PY{p}{(}\PY{l+s+s2}{\PYZdq{}}\PY{l+s+s2}{problem}\PY{l+s+s2}{\PYZdq{}}\PY{o}{+}\PY{n}{domain\PYZus{}name}\PY{o}{+}\PY{l+s+s2}{\PYZdq{}}\PY{l+s+s2}{.pddl}\PY{l+s+s2}{\PYZdq{}}\PY{p}{,}\PY{l+s+s2}{\PYZdq{}}\PY{l+s+s2}{w}\PY{l+s+s2}{\PYZdq{}}\PY{p}{)} \PY{k}{as} \PY{n}{p}\PY{p}{:}
            \PY{n}{d}\PY{o}{.}\PY{n}{write}\PY{p}{(}\PY{n}{planning}\PY{o}{.}\PY{n}{pddl\PYZus{}utility}\PY{o}{.}\PY{n}{pddl\PYZus{}domain}\PY{p}{(}\PY{n}{domain}\PY{p}{,} \PY{n}{problem}\PY{o}{.}\PY{n}{operators}\PY{p}{(}\PY{p}{)}\PY{p}{)}\PY{p}{)}
            \PY{n}{p}\PY{o}{.}\PY{n}{write}\PY{p}{(}\PY{n}{planning}\PY{o}{.}\PY{n}{pddl\PYZus{}utility}\PY{o}{.}\PY{n}{pddl\PYZus{}problem}\PY{p}{(}\PY{n}{problem}\PY{p}{)}\PY{p}{)}
\end{Verbatim}


    \begin{Verbatim}[commandchars=\\\{\}]
Domain: Hanoi
Types: \{'disk': 'object', 'peg': 'object'\}
Operators:
  - move(?what-disk,?from-object,?to-object):
    Pre: clear(?what), clear(?to), on(?what,?from), smaller(?what,?to)
    Add: clear(?from), on(?what,?to)
    Delete: clear(?to), on(?what,?from)
Init:

    \end{Verbatim}

    \begin{center}
    \adjustimage{max size={0.9\linewidth}{0.9\paperheight}}{output_3_1.pdf}
    \end{center}
    { \hspace*{\fill} \\}
    
    \begin{Verbatim}[commandchars=\\\{\}]
Goal:

    \end{Verbatim}

    \begin{center}
    \adjustimage{max size={0.9\linewidth}{0.9\paperheight}}{output_3_3.pdf}
    \end{center}
    { \hspace*{\fill} \\}
    
    \begin{Verbatim}[commandchars=\\\{\}]
Applicable actions in 1st state:
move(disk01,disk02,peg2)
move(disk01,disk02,peg3)

    \end{Verbatim}

    We can solve this problem so that we get the following result.

    \begin{Verbatim}[commandchars=\\\{\}]
{\color{incolor}In [{\color{incolor}3}]:} \PY{n}{bfs} \PY{o}{=} \PY{n}{planning}\PY{o}{.}\PY{n}{BreadthFirstSearch}\PY{p}{(}\PY{n}{problem}\PY{p}{,} \PY{n}{verbose}\PY{o}{=}\PY{l+m+mi}{1}\PY{p}{,} \PY{n}{timeout}\PY{o}{=}\PY{l+m+mf}{30.0}\PY{p}{,} \PY{n}{node\PYZus{}bound}\PY{o}{=}\PY{k+kc}{None}\PY{p}{)}
        \PY{n}{status} \PY{o}{=} \PY{n}{bfs}\PY{p}{(}\PY{p}{)}
        \PY{k}{if} \PY{n}{status} \PY{o}{==} \PY{n}{planning}\PY{o}{.}\PY{n}{search}\PY{o}{.}\PY{n}{FOUND}\PY{p}{:}
            \PY{n}{plan} \PY{o}{=} \PY{n}{bfs}\PY{o}{.}\PY{n}{plan}\PY{p}{(}\PY{p}{)}
            \PY{n}{state\PYZus{}sequence} \PY{o}{=} \PY{n}{bfs}\PY{o}{.}\PY{n}{state\PYZus{}sequence}\PY{p}{(}\PY{p}{)}
            \PY{n+nb}{print}\PY{p}{(}\PY{l+s+s2}{\PYZdq{}}\PY{l+s+s2}{\PYZsh{}Generated nodes: }\PY{l+s+s2}{\PYZdq{}} \PY{o}{+} \PY{n+nb}{str}\PY{p}{(}\PY{n}{bfs}\PY{o}{.}\PY{n}{info}\PY{p}{(}\PY{p}{)}\PY{p}{[}\PY{l+s+s2}{\PYZdq{}}\PY{l+s+s2}{generated}\PY{l+s+s2}{\PYZdq{}}\PY{p}{]}\PY{p}{)}\PY{p}{)}
            \PY{k}{if} \PY{k+kc}{True}\PY{p}{:} \PY{c+c1}{\PYZsh{}len(plan) \PYZlt{} 10:}
                \PY{n+nb}{print}\PY{p}{(}\PY{l+s+s2}{\PYZdq{}}\PY{l+s+s2}{Plan:}\PY{l+s+se}{\PYZbs{}n}\PY{l+s+s2}{  }\PY{l+s+s2}{\PYZdq{}} \PY{o}{+} \PY{l+s+s2}{\PYZdq{}}\PY{l+s+se}{\PYZbs{}n}\PY{l+s+s2}{  }\PY{l+s+s2}{\PYZdq{}}\PY{o}{.}\PY{n}{join}\PY{p}{(}\PY{p}{[}\PY{n+nb}{str}\PY{p}{(}\PY{n}{op}\PY{p}{)} \PY{k}{for} \PY{n}{op} \PY{o+ow}{in} \PY{n}{bfs}\PY{o}{.}\PY{n}{plan}\PY{p}{(}\PY{p}{)}\PY{p}{]}\PY{p}{)}\PY{p}{)}
                \PY{k}{for} \PY{n}{i}\PY{p}{,} \PY{n}{state} \PY{o+ow}{in} \PY{n+nb}{enumerate}\PY{p}{(}\PY{n}{bfs}\PY{o}{.}\PY{n}{state\PYZus{}sequence}\PY{p}{(}\PY{p}{)}\PY{p}{)}\PY{p}{:}
                    \PY{n+nb}{print}\PY{p}{(}\PY{l+s+s2}{\PYZdq{}}\PY{l+s+s2}{step }\PY{l+s+s2}{\PYZdq{}} \PY{o}{+} \PY{n+nb}{str}\PY{p}{(}\PY{n}{i}\PY{p}{)}\PY{p}{)}
                    \PY{n}{display}\PY{p}{(}\PY{n}{state}\PY{p}{)}
\end{Verbatim}


    \begin{Verbatim}[commandchars=\\\{\}]
[0.004s] Plan found with 7 action(s)
\#Generated nodes: 24
Plan:
  move(disk01,disk02,peg3)
  move(disk02,disk03,peg2)
  move(disk01,peg3,disk02)
  move(disk03,peg1,peg3)
  move(disk01,disk02,peg1)
  move(disk02,peg2,disk03)
  move(disk01,peg1,disk02)
step 0

    \end{Verbatim}

    \begin{center}
    \adjustimage{max size={0.9\linewidth}{0.9\paperheight}}{output_5_1.pdf}
    \end{center}
    { \hspace*{\fill} \\}
    
    \begin{Verbatim}[commandchars=\\\{\}]
step 1

    \end{Verbatim}

    \begin{center}
    \adjustimage{max size={0.9\linewidth}{0.9\paperheight}}{output_5_3.pdf}
    \end{center}
    { \hspace*{\fill} \\}
    
    \begin{Verbatim}[commandchars=\\\{\}]
step 2

    \end{Verbatim}

    \begin{center}
    \adjustimage{max size={0.9\linewidth}{0.9\paperheight}}{output_5_5.pdf}
    \end{center}
    { \hspace*{\fill} \\}
    
    \begin{Verbatim}[commandchars=\\\{\}]
step 3

    \end{Verbatim}

    \begin{center}
    \adjustimage{max size={0.9\linewidth}{0.9\paperheight}}{output_5_7.pdf}
    \end{center}
    { \hspace*{\fill} \\}
    
    \begin{Verbatim}[commandchars=\\\{\}]
step 4

    \end{Verbatim}

    \begin{center}
    \adjustimage{max size={0.9\linewidth}{0.9\paperheight}}{output_5_9.pdf}
    \end{center}
    { \hspace*{\fill} \\}
    
    \begin{Verbatim}[commandchars=\\\{\}]
step 5

    \end{Verbatim}

    \begin{center}
    \adjustimage{max size={0.9\linewidth}{0.9\paperheight}}{output_5_11.pdf}
    \end{center}
    { \hspace*{\fill} \\}
    
    \begin{Verbatim}[commandchars=\\\{\}]
step 6

    \end{Verbatim}

    \begin{center}
    \adjustimage{max size={0.9\linewidth}{0.9\paperheight}}{output_5_13.pdf}
    \end{center}
    { \hspace*{\fill} \\}
    
    \begin{Verbatim}[commandchars=\\\{\}]
step 7

    \end{Verbatim}

    \begin{center}
    \adjustimage{max size={0.9\linewidth}{0.9\paperheight}}{output_5_15.pdf}
    \end{center}
    { \hspace*{\fill} \\}
    
    \paragraph{1.2.2. Blocks World}\label{blocks-world}

The blocks world is one of the most famous planning domains in
artificial intelligence. Imagine a set of wooden blocks of various
shapes and colors sitting on a table. The goal is to build one or more
vertical stacks of blocks. The catch is that only one block may be moved
at a time: it may either be placed on the table or placed atop another
block. Because of this, any blocks that are, at a given time, under
another block cannot be moved. Moreover, some kinds of blocks cannot
have other blocks stacked on top of them.

The simplicity of this toy world lends itself readily to symbolic or
classical A.I. approaches, in which the world is modeled as a set of
abstract symbols which may be reasoned about.

One relevant difference between our implementation of the domain with
respect to the original one, is that our table is not infinite, we
actually let users choose the number of slots (positions) that the table
will have.

We have defined the domain in Python and it contains a two operators:
\texttt{pick(?what-block,\ ?from-object,\ ?to-object)}, which picks
block \texttt{?what} from object \texttt{?from} (either a slot of the
table or another block) and \texttt{put(?what-block,\ ?to-object)},
which puts block to object \texttt{?to} (also, either one position of
the table or a block). Our planning framework can take care of static
preconditions (no instantiations of operators for one block with
itself).

We have also implemented a problem generator. The generator can create
problems for any number of blocks and slots on the table. Our first
example will be fairly simple and will have 3 blocks, and 3 slots.

    \begin{Verbatim}[commandchars=\\\{\}]
{\color{incolor}In [{\color{incolor}4}]:} \PY{k+kn}{import} \PY{n+nn}{planning}
        
        \PY{k+kn}{from} \PY{n+nn}{IPython}\PY{n+nn}{.}\PY{n+nn}{display} \PY{k}{import} \PY{n}{display}
\end{Verbatim}


    \begin{Verbatim}[commandchars=\\\{\}]
{\color{incolor}In [{\color{incolor}5}]:} \PY{n}{domain\PYZus{}name} \PY{o}{=} \PY{l+s+s2}{\PYZdq{}}\PY{l+s+s2}{Blocks}\PY{l+s+s2}{\PYZdq{}}
        \PY{n}{domain} \PY{o}{=} \PY{n}{planning}\PY{o}{.}\PY{n}{domains}\PY{p}{[}\PY{n}{domain\PYZus{}name}\PY{p}{]}
        \PY{n}{problem} \PY{o}{=} \PY{n}{domain}\PY{o}{.}\PY{n}{generate\PYZus{}problem}\PY{p}{(}\PY{l+m+mi}{3}\PY{p}{,} \PY{l+m+mi}{3}\PY{p}{)}
        \PY{n+nb}{print}\PY{p}{(}\PY{n}{domain}\PY{p}{)}
        \PY{c+c1}{\PYZsh{} print(problem)}
        \PY{n+nb}{print}\PY{p}{(}\PY{l+s+s2}{\PYZdq{}}\PY{l+s+s2}{Init:}\PY{l+s+s2}{\PYZdq{}}\PY{p}{)}
        \PY{n}{display}\PY{p}{(}\PY{n}{problem}\PY{o}{.}\PY{n}{init}\PY{p}{(}\PY{p}{)}\PY{p}{)}
        \PY{n+nb}{print}\PY{p}{(}\PY{l+s+s2}{\PYZdq{}}\PY{l+s+s2}{Goal:}\PY{l+s+s2}{\PYZdq{}}\PY{p}{)}
        \PY{n}{display}\PY{p}{(}\PY{n}{problem}\PY{o}{.}\PY{n}{goal}\PY{p}{(}\PY{p}{)}\PY{p}{)}
        \PY{n+nb}{print}\PY{p}{(}\PY{l+s+s2}{\PYZdq{}}\PY{l+s+s2}{Applicable actions in 1st state:}\PY{l+s+s2}{\PYZdq{}}\PY{p}{)}
        \PY{n}{l} \PY{o}{=} \PY{n+nb}{list}\PY{p}{(}\PY{n+nb}{filter}\PY{p}{(}\PY{n}{problem}\PY{o}{.}\PY{n}{init}\PY{p}{(}\PY{p}{)}\PY{o}{.}\PY{n}{can\PYZus{}apply}\PY{p}{,} \PY{n}{problem}\PY{o}{.}\PY{n}{operators}\PY{p}{(}\PY{p}{)}\PY{p}{)}\PY{p}{)}
        \PY{n+nb}{print}\PY{p}{(}\PY{l+s+s2}{\PYZdq{}}\PY{l+s+se}{\PYZbs{}n}\PY{l+s+s2}{\PYZdq{}}\PY{o}{.}\PY{n}{join}\PY{p}{(}\PY{n+nb}{map}\PY{p}{(}\PY{n+nb}{str}\PY{p}{,} \PY{n}{l}\PY{p}{)}\PY{p}{)}\PY{p}{)}
        
        \PY{k}{with} \PY{n+nb}{open}\PY{p}{(}\PY{l+s+s2}{\PYZdq{}}\PY{l+s+s2}{domain}\PY{l+s+s2}{\PYZdq{}}\PY{o}{+}\PY{n}{domain\PYZus{}name}\PY{o}{+}\PY{l+s+s2}{\PYZdq{}}\PY{l+s+s2}{.pddl}\PY{l+s+s2}{\PYZdq{}}\PY{p}{,} \PY{l+s+s2}{\PYZdq{}}\PY{l+s+s2}{w}\PY{l+s+s2}{\PYZdq{}}\PY{p}{)} \PY{k}{as} \PY{n}{d}\PY{p}{,} \PY{n+nb}{open}\PY{p}{(}\PY{l+s+s2}{\PYZdq{}}\PY{l+s+s2}{problem}\PY{l+s+s2}{\PYZdq{}}\PY{o}{+}\PY{n}{domain\PYZus{}name}\PY{o}{+}\PY{l+s+s2}{\PYZdq{}}\PY{l+s+s2}{.pddl}\PY{l+s+s2}{\PYZdq{}}\PY{p}{,}\PY{l+s+s2}{\PYZdq{}}\PY{l+s+s2}{w}\PY{l+s+s2}{\PYZdq{}}\PY{p}{)} \PY{k}{as} \PY{n}{p}\PY{p}{:}
            \PY{n}{d}\PY{o}{.}\PY{n}{write}\PY{p}{(}\PY{n}{planning}\PY{o}{.}\PY{n}{pddl\PYZus{}utility}\PY{o}{.}\PY{n}{pddl\PYZus{}domain}\PY{p}{(}\PY{n}{domain}\PY{p}{,} \PY{n}{problem}\PY{o}{.}\PY{n}{operators}\PY{p}{(}\PY{p}{)}\PY{p}{)}\PY{p}{)}
            \PY{n}{p}\PY{o}{.}\PY{n}{write}\PY{p}{(}\PY{n}{planning}\PY{o}{.}\PY{n}{pddl\PYZus{}utility}\PY{o}{.}\PY{n}{pddl\PYZus{}problem}\PY{p}{(}\PY{n}{problem}\PY{p}{)}\PY{p}{)}
\end{Verbatim}


    \begin{Verbatim}[commandchars=\\\{\}]
Domain: Blocks
Types: \{'block': 'object', 'slot': 'object'\}
Operators:
  - pick(?what-block,?from-object):
    Pre: clear(?what), on(?what,?from), handempty()
    Add: clear(?from), holding(?what)
    Delete: clear(?what), on(?what,?from), handempty()
  - put(?what-block,?to-object):
    Pre: holding(?what), clear(?to)
    Add: clear(?what), on(?what,?to), handempty()
    Delete: holding(?what), clear(?to)
Init:

    \end{Verbatim}

    \begin{center}
    \adjustimage{max size={0.9\linewidth}{0.9\paperheight}}{output_8_1.pdf}
    \end{center}
    { \hspace*{\fill} \\}
    
    \begin{Verbatim}[commandchars=\\\{\}]
Goal:

    \end{Verbatim}

    \begin{center}
    \adjustimage{max size={0.9\linewidth}{0.9\paperheight}}{output_8_3.pdf}
    \end{center}
    { \hspace*{\fill} \\}
    
    \begin{Verbatim}[commandchars=\\\{\}]
Applicable actions in 1st state:
pick(block01,slot3)
pick(block02,block03)

    \end{Verbatim}

    We can solve this problem so that we get the following result.

    \begin{Verbatim}[commandchars=\\\{\}]
{\color{incolor}In [{\color{incolor}6}]:} \PY{n}{bfs} \PY{o}{=} \PY{n}{planning}\PY{o}{.}\PY{n}{BreadthFirstSearch}\PY{p}{(}\PY{n}{problem}\PY{p}{,} \PY{n}{verbose}\PY{o}{=}\PY{l+m+mi}{1}\PY{p}{,} \PY{n}{timeout}\PY{o}{=}\PY{l+m+mf}{30.0}\PY{p}{,} \PY{n}{node\PYZus{}bound}\PY{o}{=}\PY{k+kc}{None}\PY{p}{)}
        \PY{n}{status} \PY{o}{=} \PY{n}{bfs}\PY{p}{(}\PY{p}{)}
        \PY{k}{if} \PY{n}{status} \PY{o}{==} \PY{n}{planning}\PY{o}{.}\PY{n}{search}\PY{o}{.}\PY{n}{FOUND}\PY{p}{:}
            \PY{n}{plan} \PY{o}{=} \PY{n}{bfs}\PY{o}{.}\PY{n}{plan}\PY{p}{(}\PY{p}{)}
            \PY{n}{state\PYZus{}sequence} \PY{o}{=} \PY{n}{bfs}\PY{o}{.}\PY{n}{state\PYZus{}sequence}\PY{p}{(}\PY{p}{)}
            \PY{n+nb}{print}\PY{p}{(}\PY{l+s+s2}{\PYZdq{}}\PY{l+s+s2}{\PYZsh{}Generated nodes: }\PY{l+s+s2}{\PYZdq{}} \PY{o}{+} \PY{n+nb}{str}\PY{p}{(}\PY{n}{bfs}\PY{o}{.}\PY{n}{info}\PY{p}{(}\PY{p}{)}\PY{p}{[}\PY{l+s+s2}{\PYZdq{}}\PY{l+s+s2}{generated}\PY{l+s+s2}{\PYZdq{}}\PY{p}{]}\PY{p}{)}\PY{p}{)}
            \PY{k}{if} \PY{k+kc}{True}\PY{p}{:} \PY{c+c1}{\PYZsh{}len(plan) \PYZlt{} 10:}
                \PY{n+nb}{print}\PY{p}{(}\PY{l+s+s2}{\PYZdq{}}\PY{l+s+s2}{Plan:}\PY{l+s+se}{\PYZbs{}n}\PY{l+s+s2}{  }\PY{l+s+s2}{\PYZdq{}} \PY{o}{+} \PY{l+s+s2}{\PYZdq{}}\PY{l+s+se}{\PYZbs{}n}\PY{l+s+s2}{  }\PY{l+s+s2}{\PYZdq{}}\PY{o}{.}\PY{n}{join}\PY{p}{(}\PY{p}{[}\PY{n+nb}{str}\PY{p}{(}\PY{n}{op}\PY{p}{)} \PY{k}{for} \PY{n}{op} \PY{o+ow}{in} \PY{n}{bfs}\PY{o}{.}\PY{n}{plan}\PY{p}{(}\PY{p}{)}\PY{p}{]}\PY{p}{)}\PY{p}{)}
                \PY{k}{for} \PY{n}{i}\PY{p}{,} \PY{n}{state} \PY{o+ow}{in} \PY{n+nb}{enumerate}\PY{p}{(}\PY{n}{bfs}\PY{o}{.}\PY{n}{state\PYZus{}sequence}\PY{p}{(}\PY{p}{)}\PY{p}{)}\PY{p}{:}
                    \PY{n+nb}{print}\PY{p}{(}\PY{l+s+s2}{\PYZdq{}}\PY{l+s+s2}{step }\PY{l+s+s2}{\PYZdq{}} \PY{o}{+} \PY{n+nb}{str}\PY{p}{(}\PY{n}{i}\PY{p}{)}\PY{p}{)}
                    \PY{n}{display}\PY{p}{(}\PY{n}{state}\PY{p}{)}
\end{Verbatim}


    \begin{Verbatim}[commandchars=\\\{\}]
[0.011s] Plan found with 10 action(s)
\#Generated nodes: 85
Plan:
  pick(block01,slot3)
  put(block01,slot1)
  pick(block02,block03)
  put(block02,block01)
  pick(block03,slot2)
  put(block03,slot3)
  pick(block02,block01)
  put(block02,block03)
  pick(block01,slot1)
  put(block01,slot2)
step 0

    \end{Verbatim}

    \begin{center}
    \adjustimage{max size={0.9\linewidth}{0.9\paperheight}}{output_10_1.pdf}
    \end{center}
    { \hspace*{\fill} \\}
    
    \begin{Verbatim}[commandchars=\\\{\}]
step 1

    \end{Verbatim}

    \begin{center}
    \adjustimage{max size={0.9\linewidth}{0.9\paperheight}}{output_10_3.pdf}
    \end{center}
    { \hspace*{\fill} \\}
    
    \begin{Verbatim}[commandchars=\\\{\}]
step 2

    \end{Verbatim}

    \begin{center}
    \adjustimage{max size={0.9\linewidth}{0.9\paperheight}}{output_10_5.pdf}
    \end{center}
    { \hspace*{\fill} \\}
    
    \begin{Verbatim}[commandchars=\\\{\}]
step 3

    \end{Verbatim}

    \begin{center}
    \adjustimage{max size={0.9\linewidth}{0.9\paperheight}}{output_10_7.pdf}
    \end{center}
    { \hspace*{\fill} \\}
    
    \begin{Verbatim}[commandchars=\\\{\}]
step 4

    \end{Verbatim}

    \begin{center}
    \adjustimage{max size={0.9\linewidth}{0.9\paperheight}}{output_10_9.pdf}
    \end{center}
    { \hspace*{\fill} \\}
    
    \begin{Verbatim}[commandchars=\\\{\}]
step 5

    \end{Verbatim}

    \begin{center}
    \adjustimage{max size={0.9\linewidth}{0.9\paperheight}}{output_10_11.pdf}
    \end{center}
    { \hspace*{\fill} \\}
    
    \begin{Verbatim}[commandchars=\\\{\}]
step 6

    \end{Verbatim}

    \begin{center}
    \adjustimage{max size={0.9\linewidth}{0.9\paperheight}}{output_10_13.pdf}
    \end{center}
    { \hspace*{\fill} \\}
    
    \begin{Verbatim}[commandchars=\\\{\}]
step 7

    \end{Verbatim}

    \begin{center}
    \adjustimage{max size={0.9\linewidth}{0.9\paperheight}}{output_10_15.pdf}
    \end{center}
    { \hspace*{\fill} \\}
    
    \begin{Verbatim}[commandchars=\\\{\}]
step 8

    \end{Verbatim}

    \begin{center}
    \adjustimage{max size={0.9\linewidth}{0.9\paperheight}}{output_10_17.pdf}
    \end{center}
    { \hspace*{\fill} \\}
    
    \begin{Verbatim}[commandchars=\\\{\}]
step 9

    \end{Verbatim}

    \begin{center}
    \adjustimage{max size={0.9\linewidth}{0.9\paperheight}}{output_10_19.pdf}
    \end{center}
    { \hspace*{\fill} \\}
    
    \begin{Verbatim}[commandchars=\\\{\}]
step 10

    \end{Verbatim}

    \begin{center}
    \adjustimage{max size={0.9\linewidth}{0.9\paperheight}}{output_10_21.pdf}
    \end{center}
    { \hspace*{\fill} \\}
    
    \paragraph{1.2.3. Elevators Domain}\label{elevators-domain}

Let's consider we have a building with several foors and people who want
to go from one floor to another one. A lift, starting from one of the
floors, has to satisfy all people petitions. The resctrictions are the
following:

\begin{enumerate}
\def\labelenumi{\arabic{enumi}.}
\tightlist
\item
  The lift can just move one floor each step
\item
  Just one person can get into/outo the lift each step
\end{enumerate}

We have defined the domain in Python and it contains a four operators:

\begin{enumerate}
\def\labelenumi{\arabic{enumi}.}
\tightlist
\item
  \texttt{board(?f-floor,\ ?p-passenger)}, which boards the passenger
  \texttt{?p} which is in the floor \texttt{?f}.
\item
  \texttt{depart(?f-floor,\ ?p-passenger)}, which departs the passenger
  \texttt{?p} in the floor \texttt{?f}. The passenger passes to be
  served.
\item
  \texttt{drive\_up(?f1-floor,\ ?f2-floor)}, the lift goes up from floor
  \texttt{?f1} to floor \texttt{?f2}.
\item
  \texttt{drive\_down(?f1-floor,\ ?f2-floor)}, the lift goes down from
  floor \texttt{?f1} to floor \texttt{?f2}.
\end{enumerate}

Our planning framework can take care of static preconditions (no
instantiations of operators for the same instance).

We have also implemented a problem generator. The generator can create
problems for any number of people and floors. Our first example will be
fairly simple and will have 4 people and 4 floors. Again, we use BFS and
IDS.

    \begin{Verbatim}[commandchars=\\\{\}]
{\color{incolor}In [{\color{incolor}7}]:} \PY{n}{domain\PYZus{}name} \PY{o}{=} \PY{l+s+s2}{\PYZdq{}}\PY{l+s+s2}{Elevator}\PY{l+s+s2}{\PYZdq{}}
        \PY{n}{domain} \PY{o}{=} \PY{n}{planning}\PY{o}{.}\PY{n}{domains}\PY{p}{[}\PY{n}{domain\PYZus{}name}\PY{p}{]}
        \PY{n}{problem} \PY{o}{=} \PY{n}{domain}\PY{o}{.}\PY{n}{generate\PYZus{}problem}\PY{p}{(}\PY{l+m+mi}{3}\PY{p}{,} \PY{l+m+mi}{4}\PY{p}{)}
        \PY{n+nb}{print}\PY{p}{(}\PY{n}{domain}\PY{p}{)}
        \PY{n+nb}{print}\PY{p}{(}\PY{n}{problem}\PY{p}{)}
        \PY{n+nb}{print}\PY{p}{(}\PY{l+s+s2}{\PYZdq{}}\PY{l+s+s2}{Init:}\PY{l+s+s2}{\PYZdq{}}\PY{p}{)}
        \PY{n}{display}\PY{p}{(}\PY{n}{problem}\PY{o}{.}\PY{n}{init}\PY{p}{(}\PY{p}{)}\PY{p}{)}
        \PY{n+nb}{print}\PY{p}{(}\PY{l+s+s2}{\PYZdq{}}\PY{l+s+s2}{Goal:}\PY{l+s+s2}{\PYZdq{}}\PY{p}{)}
        \PY{c+c1}{\PYZsh{}display(problem.goal())}
        \PY{n+nb}{print}\PY{p}{(}\PY{l+s+s2}{\PYZdq{}}\PY{l+s+s2}{Applicable actions in 1st state:}\PY{l+s+s2}{\PYZdq{}}\PY{p}{)}
        \PY{n+nb}{print}\PY{p}{(}\PY{l+s+s2}{\PYZdq{}}\PY{l+s+se}{\PYZbs{}n}\PY{l+s+s2}{\PYZdq{}}\PY{o}{.}\PY{n}{join}\PY{p}{(}\PY{n+nb}{map}\PY{p}{(}\PY{n+nb}{str}\PY{p}{,} \PY{n+nb}{filter}\PY{p}{(}\PY{n}{problem}\PY{o}{.}\PY{n}{init}\PY{p}{(}\PY{p}{)}\PY{o}{.}\PY{n}{can\PYZus{}apply}\PY{p}{,} \PY{n}{problem}\PY{o}{.}\PY{n}{operators}\PY{p}{(}\PY{p}{)}\PY{p}{)}\PY{p}{)}\PY{p}{)}\PY{p}{)}
        
        \PY{k}{with} \PY{n+nb}{open}\PY{p}{(}\PY{l+s+s2}{\PYZdq{}}\PY{l+s+s2}{domain}\PY{l+s+s2}{\PYZdq{}}\PY{o}{+}\PY{n}{domain\PYZus{}name}\PY{o}{+}\PY{l+s+s2}{\PYZdq{}}\PY{l+s+s2}{.pddl}\PY{l+s+s2}{\PYZdq{}}\PY{p}{,} \PY{l+s+s2}{\PYZdq{}}\PY{l+s+s2}{w}\PY{l+s+s2}{\PYZdq{}}\PY{p}{)} \PY{k}{as} \PY{n}{d}\PY{p}{,} \PY{n+nb}{open}\PY{p}{(}\PY{l+s+s2}{\PYZdq{}}\PY{l+s+s2}{problem}\PY{l+s+s2}{\PYZdq{}}\PY{o}{+}\PY{n}{domain\PYZus{}name}\PY{o}{+}\PY{l+s+s2}{\PYZdq{}}\PY{l+s+s2}{.pddl}\PY{l+s+s2}{\PYZdq{}}\PY{p}{,}\PY{l+s+s2}{\PYZdq{}}\PY{l+s+s2}{w}\PY{l+s+s2}{\PYZdq{}}\PY{p}{)} \PY{k}{as} \PY{n}{p}\PY{p}{:}
            \PY{n}{d}\PY{o}{.}\PY{n}{write}\PY{p}{(}\PY{n}{planning}\PY{o}{.}\PY{n}{pddl\PYZus{}utility}\PY{o}{.}\PY{n}{pddl\PYZus{}domain}\PY{p}{(}\PY{n}{domain}\PY{p}{,} \PY{n}{problem}\PY{o}{.}\PY{n}{operators}\PY{p}{(}\PY{p}{)}\PY{p}{)}\PY{p}{)}
            \PY{n}{p}\PY{o}{.}\PY{n}{write}\PY{p}{(}\PY{n}{planning}\PY{o}{.}\PY{n}{pddl\PYZus{}utility}\PY{o}{.}\PY{n}{pddl\PYZus{}problem}\PY{p}{(}\PY{n}{problem}\PY{p}{)}\PY{p}{)}
\end{Verbatim}


    \begin{Verbatim}[commandchars=\\\{\}]
Domain: Elevator
Types: \{'passenger': 'object', 'floor': 'object'\}
Operators:
  - board(?f-floor,?p-passenger):
    Pre: lift-at(?f), origin(?p,?f)
    Add: boarded(?p)
    Delete: 
  - depart(?f-floor,?p-passenger):
    Pre: lift-at(?f), destin(?p,?f), boarded(?p)
    Add: served(?p)
    Delete: boarded(?p)
  - drive\_down(?f1-floor,?f2-floor):
    Pre: lift-at(?f1), above(?f2,?f1)
    Add: lift-at(?f2)
    Delete: lift-at(?f1)
  - drive\_up(?f1-floor,?f2-floor):
    Pre: lift-at(?f1), above(?f1,?f2)
    Add: lift-at(?f2)
    Delete: lift-at(?f1)
Problem: elevator-00
Domain: Elevator
Initial state:
  above(floor1,floor2)
  above(floor2,floor3)
  above(floor3,floor4)
  destin(passenger01,floor4)
  destin(passenger02,floor1)
  destin(passenger03,floor4)
  lift-at(floor1)
  origin(passenger01,floor3)
  origin(passenger02,floor2)
  origin(passenger03,floor2)
Goal:
  above(floor1,floor2)
  above(floor2,floor3)
  above(floor3,floor4)
  destin(passenger01,floor4)
  destin(passenger02,floor1)
  destin(passenger03,floor4)
  origin(passenger01,floor3)
  origin(passenger02,floor2)
  origin(passenger03,floor2)
  served(passenger01)
  served(passenger02)
  served(passenger03)
Ground operators:
  board(floor1,passenger01)
  board(floor1,passenger02)
  board(floor1,passenger03)
  board(floor2,passenger01)
  board(floor2,passenger02)
  board(floor2,passenger03)
  board(floor3,passenger01)
  board(floor3,passenger02)
  board(floor3,passenger03)
  board(floor4,passenger01)
  board(floor4,passenger02)
  board(floor4,passenger03)
  depart(floor1,passenger01)
  depart(floor1,passenger02)
  depart(floor1,passenger03)
  depart(floor2,passenger01)
  depart(floor2,passenger02)
  depart(floor2,passenger03)
  depart(floor3,passenger01)
  depart(floor3,passenger02)
  depart(floor3,passenger03)
  depart(floor4,passenger01)
  depart(floor4,passenger02)
  depart(floor4,passenger03)
  drive\_down(floor1,floor2)
  drive\_down(floor1,floor3)
  drive\_down(floor1,floor4)
  drive\_down(floor2,floor1)
  drive\_down(floor2,floor3)
  drive\_down(floor2,floor4)
  drive\_down(floor3,floor1)
  drive\_down(floor3,floor2)
  drive\_down(floor3,floor4)
  drive\_down(floor4,floor1)
  drive\_down(floor4,floor2)
  drive\_down(floor4,floor3)
  drive\_up(floor1,floor2)
  drive\_up(floor1,floor3)
  drive\_up(floor1,floor4)
  drive\_up(floor2,floor1)
  drive\_up(floor2,floor3)
  drive\_up(floor2,floor4)
  drive\_up(floor3,floor1)
  drive\_up(floor3,floor2)
  drive\_up(floor3,floor4)
  drive\_up(floor4,floor1)
  drive\_up(floor4,floor2)
  drive\_up(floor4,floor3)
Init:
\{'floor4': 0, 'floor3': 1, 'floor1': 0, 'floor2': 2\}

    \end{Verbatim}

    \begin{center}
    \adjustimage{max size={0.9\linewidth}{0.9\paperheight}}{output_12_1.pdf}
    \end{center}
    { \hspace*{\fill} \\}
    
    \begin{Verbatim}[commandchars=\\\{\}]
Goal:
Applicable actions in 1st state:
drive\_up(floor1,floor2)

    \end{Verbatim}

    We can solve this problem so that we get the following result.

    \begin{Verbatim}[commandchars=\\\{\}]
{\color{incolor}In [{\color{incolor}8}]:} \PY{n}{bfs} \PY{o}{=} \PY{n}{planning}\PY{o}{.}\PY{n}{BreadthFirstSearch}\PY{p}{(}\PY{n}{problem}\PY{p}{,} \PY{n}{verbose}\PY{o}{=}\PY{l+m+mi}{1}\PY{p}{,} \PY{n}{timeout}\PY{o}{=}\PY{l+m+mf}{30.0}\PY{p}{,} \PY{n}{node\PYZus{}bound}\PY{o}{=}\PY{k+kc}{None}\PY{p}{)}
        \PY{n}{status} \PY{o}{=} \PY{n}{bfs}\PY{p}{(}\PY{p}{)}
        \PY{k}{if} \PY{n}{status} \PY{o}{==} \PY{n}{planning}\PY{o}{.}\PY{n}{search}\PY{o}{.}\PY{n}{FOUND}\PY{p}{:}
            \PY{n}{plan} \PY{o}{=} \PY{n}{bfs}\PY{o}{.}\PY{n}{plan}\PY{p}{(}\PY{p}{)}
            \PY{n}{state\PYZus{}sequence} \PY{o}{=} \PY{n}{bfs}\PY{o}{.}\PY{n}{state\PYZus{}sequence}\PY{p}{(}\PY{p}{)}
            \PY{n+nb}{print}\PY{p}{(}\PY{l+s+s2}{\PYZdq{}}\PY{l+s+s2}{\PYZsh{}Generated nodes: }\PY{l+s+s2}{\PYZdq{}} \PY{o}{+} \PY{n+nb}{str}\PY{p}{(}\PY{n}{bfs}\PY{o}{.}\PY{n}{info}\PY{p}{(}\PY{p}{)}\PY{p}{[}\PY{l+s+s2}{\PYZdq{}}\PY{l+s+s2}{generated}\PY{l+s+s2}{\PYZdq{}}\PY{p}{]}\PY{p}{)}\PY{p}{)}
            \PY{k}{if} \PY{k+kc}{True}\PY{p}{:} \PY{c+c1}{\PYZsh{}len(plan) \PYZlt{} 10:}
                \PY{n+nb}{print}\PY{p}{(}\PY{l+s+s2}{\PYZdq{}}\PY{l+s+s2}{Plan:}\PY{l+s+se}{\PYZbs{}n}\PY{l+s+s2}{  }\PY{l+s+s2}{\PYZdq{}} \PY{o}{+} \PY{l+s+s2}{\PYZdq{}}\PY{l+s+se}{\PYZbs{}n}\PY{l+s+s2}{  }\PY{l+s+s2}{\PYZdq{}}\PY{o}{.}\PY{n}{join}\PY{p}{(}\PY{p}{[}\PY{n+nb}{str}\PY{p}{(}\PY{n}{op}\PY{p}{)} \PY{k}{for} \PY{n}{op} \PY{o+ow}{in} \PY{n}{bfs}\PY{o}{.}\PY{n}{plan}\PY{p}{(}\PY{p}{)}\PY{p}{]}\PY{p}{)}\PY{p}{)}
                \PY{k}{for} \PY{n}{i}\PY{p}{,} \PY{n}{state} \PY{o+ow}{in} \PY{n+nb}{enumerate}\PY{p}{(}\PY{n}{bfs}\PY{o}{.}\PY{n}{state\PYZus{}sequence}\PY{p}{(}\PY{p}{)}\PY{p}{)}\PY{p}{:}
                    \PY{n+nb}{print}\PY{p}{(}\PY{l+s+s2}{\PYZdq{}}\PY{l+s+s2}{step }\PY{l+s+s2}{\PYZdq{}} \PY{o}{+} \PY{n+nb}{str}\PY{p}{(}\PY{n}{i}\PY{p}{)}\PY{p}{)}
                    \PY{n}{display}\PY{p}{(}\PY{n}{state}\PY{p}{)}
\end{Verbatim}


    \begin{Verbatim}[commandchars=\\\{\}]
[0.018s] Plan found with 11 action(s)
\#Generated nodes: 147
Plan:
  drive\_up(floor1,floor2)
  board(floor2,passenger02)
  board(floor2,passenger03)
  drive\_down(floor2,floor1)
  depart(floor1,passenger02)
  drive\_up(floor1,floor2)
  drive\_up(floor2,floor3)
  board(floor3,passenger01)
  drive\_up(floor3,floor4)
  depart(floor4,passenger01)
  depart(floor4,passenger03)
step 0
\{'floor4': 0, 'floor3': 1, 'floor1': 0, 'floor2': 2\}

    \end{Verbatim}

    \begin{center}
    \adjustimage{max size={0.9\linewidth}{0.9\paperheight}}{output_14_1.pdf}
    \end{center}
    { \hspace*{\fill} \\}
    
    \begin{Verbatim}[commandchars=\\\{\}]
step 1
\{'floor4': 0, 'floor3': 1, 'floor1': 0, 'floor2': 2\}

    \end{Verbatim}

    \begin{center}
    \adjustimage{max size={0.9\linewidth}{0.9\paperheight}}{output_14_3.pdf}
    \end{center}
    { \hspace*{\fill} \\}
    
    \begin{Verbatim}[commandchars=\\\{\}]
step 2
\{'floor4': 0, 'floor3': 1, 'floor1': 0, 'floor2': 1\}

    \end{Verbatim}

    \begin{center}
    \adjustimage{max size={0.9\linewidth}{0.9\paperheight}}{output_14_5.pdf}
    \end{center}
    { \hspace*{\fill} \\}
    
    \begin{Verbatim}[commandchars=\\\{\}]
step 3
\{'floor4': 0, 'floor3': 1, 'floor1': 0, 'floor2': 0\}

    \end{Verbatim}

    \begin{center}
    \adjustimage{max size={0.9\linewidth}{0.9\paperheight}}{output_14_7.pdf}
    \end{center}
    { \hspace*{\fill} \\}
    
    \begin{Verbatim}[commandchars=\\\{\}]
step 4
\{'floor4': 0, 'floor3': 1, 'floor1': 0, 'floor2': 0\}

    \end{Verbatim}

    \begin{center}
    \adjustimage{max size={0.9\linewidth}{0.9\paperheight}}{output_14_9.pdf}
    \end{center}
    { \hspace*{\fill} \\}
    
    \begin{Verbatim}[commandchars=\\\{\}]
step 5
\{'floor4': 1, 'floor3': 0, 'floor1': 1, 'floor2': 0\}

    \end{Verbatim}

    \begin{center}
    \adjustimage{max size={0.9\linewidth}{0.9\paperheight}}{output_14_11.pdf}
    \end{center}
    { \hspace*{\fill} \\}
    
    \begin{Verbatim}[commandchars=\\\{\}]
step 6
\{'floor4': 1, 'floor3': 0, 'floor1': 1, 'floor2': 0\}

    \end{Verbatim}

    \begin{center}
    \adjustimage{max size={0.9\linewidth}{0.9\paperheight}}{output_14_13.pdf}
    \end{center}
    { \hspace*{\fill} \\}
    
    \begin{Verbatim}[commandchars=\\\{\}]
step 7
\{'floor4': 1, 'floor3': 0, 'floor1': 1, 'floor2': 0\}

    \end{Verbatim}

    \begin{center}
    \adjustimage{max size={0.9\linewidth}{0.9\paperheight}}{output_14_15.pdf}
    \end{center}
    { \hspace*{\fill} \\}
    
    \begin{Verbatim}[commandchars=\\\{\}]
step 8
\{'floor4': 0, 'floor3': 0, 'floor1': 1, 'floor2': 0\}

    \end{Verbatim}

    \begin{center}
    \adjustimage{max size={0.9\linewidth}{0.9\paperheight}}{output_14_17.pdf}
    \end{center}
    { \hspace*{\fill} \\}
    
    \begin{Verbatim}[commandchars=\\\{\}]
step 9
\{'floor4': 0, 'floor3': 0, 'floor1': 1, 'floor2': 0\}

    \end{Verbatim}

    \begin{center}
    \adjustimage{max size={0.9\linewidth}{0.9\paperheight}}{output_14_19.pdf}
    \end{center}
    { \hspace*{\fill} \\}
    
    \begin{Verbatim}[commandchars=\\\{\}]
step 10
\{'floor4': 1, 'floor3': 0, 'floor1': 1, 'floor2': 0\}

    \end{Verbatim}

    \begin{center}
    \adjustimage{max size={0.9\linewidth}{0.9\paperheight}}{output_14_21.pdf}
    \end{center}
    { \hspace*{\fill} \\}
    
    \begin{Verbatim}[commandchars=\\\{\}]
step 11
\{'floor4': 2, 'floor3': 0, 'floor1': 1, 'floor2': 0\}

    \end{Verbatim}

    \begin{center}
    \adjustimage{max size={0.9\linewidth}{0.9\paperheight}}{output_14_23.pdf}
    \end{center}
    { \hspace*{\fill} \\}
    
    \subsection{2. Requirement Analysis of the CBR engine
Project}\label{requirement-analysis-of-the-cbr-engine-project}

In this section we analayse both, the requirements from a user
perspective and from a technical point of view.

\subsubsection{2.1. Requirements: Users's
perspective}\label{requirements-userss-perspective}

Our system, general as it is, can be used in any sort of application in
which is interesting the use of a planning system. In any domain in
which a user has a problem which can be solved with planning techniques,
our project is useful, so that, there are no requirements from the
perspective of potential users. They will depend on the specific
application of use.

Note that the work developed in this project could be used by
specialized people with technical knowledge and this include both,
researchers and product designers at private companies.

Among the possible domains in which our system could be used we find:
robotics, industrial organization, assembly lines or applications
related to the use of satellites.

\subsubsection{2.2. Requirements: Technical
view}\label{requirements-technical-view}

As it is explained just above, our system is too general to present
requirements as specific as: maximum time response of the system or
maximum memory size of the system. All these requirements will show up
once the our potential users start using our system in a specific
application domain.

When designing our system we tried to address the following specific
requirements:

\begin{enumerate}
\def\labelenumi{\arabic{enumi}.}
\item
  The system should be faster than a conventional planner and expand
  less nodes when searching the solution.
\item
  The computed heuristic should be as precise as possible.
\item
  The CBR system was thought to start learning with simple problems of
  the specific domain and continue doing so with more difficult and
  specific problems.
\end{enumerate}

    \subsection{3. Functional Architecture of the CBR engine
Project}\label{functional-architecture-of-the-cbr-engine-project}

In this project we have designed a system of general purpose, so that
the functional architecture is quite simple to be used. Just four
different entities take part in our Functional Architecture:

\begin{itemize}
\tightlist
\item
  \textbf{Planner:} It is our system itself, a CBR trained for planning
  purposes.
\item
  \textbf{Domain:} The domain has to be coded (just once) by the user.
  In order to do so, it is necessary to follow the same syntaxis we have
  used for our three domains, which is quite intuitive and tries to be
  as close as possible to PDDL. This is one of the inputs of our CBR
  system.
\item
  \textbf{Problem:} Again, the problem has to be written by the user and
  it will be another input of our Planner.
\item
  \textbf{Plan:} This is the output of our system, the final plan with
  the actions to be performed in order to solve the input problem.
\end{itemize}

Below, we can see an image of how this four elements interact.

    \subsection{4. Proposed CBR engine Project solution
design:}\label{proposed-cbr-engine-project-solution-design}

In this section we analyse in depth the design of our CBR engine: the
Case Structure and Case Library Structure and the methods used for every
CBR cycle step.

\subsubsection{4.1. Case Structure and Case Library Structure
designed}\label{case-structure-and-case-library-structure-designed}

As said in the introduction of this document, our cases are vectors of
heuristics. Specificially, we are using eight different heuristics which
are computed using an implementation in C of Fast Downward. Fast
Downward is a classical planning system based on heuristic search. It
can deal with general deterministic planning problems encoded in the
propositional fragment of PDDL2.2, including advanced features like ADL
conditions and effects and derived predicates (axioms).

\textbf{put reference for FD...M. Helmert (2006) "The Fast Downward
Planning System", Volume 26, pages 191-246}

The implementation of Fast Downward includes several heuristics, each of
them has four different characteristics we have used to choose them:

\begin{itemize}
\item
  Admissible: h(s) \textless{}= h*(s) for all states s
\item
  Consistent: h(s) \textless{}= c(s, s') + h(s') for all states s
  connected to states s' by an action with cost c(s, s')
\item
  Safe: h(s) = infinity is only true for states with h*(s) = infinity
\item
  Preferred operators: this heuristic identifies preferred operators
\end{itemize}

From all heuristics provided by the cited implementation, we have chosen
the following eight to conform our cases vectors:

\begin{itemize}
\item
  Additive (add)
\item
  Additive CEGAR (cegar)
\item
  Causal Graph (cg)
\item
  Canonical PDB (cpdbs)
\item
  FF (ff)
\item
  Global Count (globalcount)
\item
  Max Heuristic (hmax)
\item
  Landmark Cut (lmcut)
\end{itemize}

The computation of all these heuristics increases the time of execution
of our code, since it is necesary to call C code from Python. That is
one of the reasons why we decided to implement some of the heuristics in
Python. However, we just have time to do so with the ones we identify as
fast and good enough: Additive and Relaxed Planning Graph.

Regarding the Case Library Structure, we have used \emph{k-d tree},
which is useful for recover the sublinear time in the most similar
cases, in our case, the vector of heuristics. The value of 'k'
corresponds to the number of heuristics we use, eitght.

\subsubsection{4.2. Methods of every CBR cycle
step}\label{methods-of-every-cbr-cycle-step}

Case-based reasoning has been formalized for purposes of computer
reasoning as a four-step process:

\begin{itemize}
\item
  \textbf{Retrieve:} Given a target problem, retrieve from memory cases
  relevant to solving it. In our case, we select the retrieved solution
  using the K-Nearest Neighbors (KNN) algorithm. The value of K has been
  chosen empirically after some experiments. This election should be
  improved in future versions of our system.
\item
  \textbf{Reuse (Adaptation):} Map the solution from the previous case
  to the target problem. This may involve adapting the solution as
  needed to fit the new situation. In our case we do not adapt the
  retrieved solution, of course, a more intelligent behaviour should be
  implemented in future work.
\item
  \textbf{Revise:} Having mapped the previous solution to the target
  situation, test the new solution in the real world (or a simulation)
  and, if necessary, revise. We compare our solution to the solution
  given by the A* algorithm in the case of using just one heuristic, the
  additive. We have chosen that heuristic because it works pretty well
  and is faster. Those are some of the reasons why we directly have
  implemented it in Python.
\item
  \textbf{Retain:} After the solution has been successfully adapted to
  the target problem, store the resulting experience as a new case in
  memory. We have decided to learn every plan we generate, of course, we
  could improve this with a smarter strategy.
\end{itemize}

    \subsection{5. Testing}\label{testing}

    \subsection{6. Evaluation}\label{evaluation}


    % Add a bibliography block to the postdoc
    
    
    
    \end{document}
